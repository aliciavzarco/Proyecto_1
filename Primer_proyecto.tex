\documentclass[]{article}
\usepackage{lmodern}
\usepackage{amssymb,amsmath}
\usepackage{ifxetex,ifluatex}
\usepackage{fixltx2e} % provides \textsubscript
\ifnum 0\ifxetex 1\fi\ifluatex 1\fi=0 % if pdftex
  \usepackage[T1]{fontenc}
  \usepackage[utf8]{inputenc}
\else % if luatex or xelatex
  \ifxetex
    \usepackage{mathspec}
  \else
    \usepackage{fontspec}
  \fi
  \defaultfontfeatures{Ligatures=TeX,Scale=MatchLowercase}
\fi
% use upquote if available, for straight quotes in verbatim environments
\IfFileExists{upquote.sty}{\usepackage{upquote}}{}
% use microtype if available
\IfFileExists{microtype.sty}{%
\usepackage{microtype}
\UseMicrotypeSet[protrusion]{basicmath} % disable protrusion for tt fonts
}{}
\usepackage[margin=1in]{geometry}
\usepackage{hyperref}
\hypersetup{unicode=true,
            pdftitle={ProyectoRM\_1},
            pdfauthor={Alicia Viñas Zarco},
            pdfborder={0 0 0},
            breaklinks=true}
\urlstyle{same}  % don't use monospace font for urls
\usepackage{longtable,booktabs}
\usepackage{graphicx,grffile}
\makeatletter
\def\maxwidth{\ifdim\Gin@nat@width>\linewidth\linewidth\else\Gin@nat@width\fi}
\def\maxheight{\ifdim\Gin@nat@height>\textheight\textheight\else\Gin@nat@height\fi}
\makeatother
% Scale images if necessary, so that they will not overflow the page
% margins by default, and it is still possible to overwrite the defaults
% using explicit options in \includegraphics[width, height, ...]{}
\setkeys{Gin}{width=\maxwidth,height=\maxheight,keepaspectratio}
\IfFileExists{parskip.sty}{%
\usepackage{parskip}
}{% else
\setlength{\parindent}{0pt}
\setlength{\parskip}{6pt plus 2pt minus 1pt}
}
\setlength{\emergencystretch}{3em}  % prevent overfull lines
\providecommand{\tightlist}{%
  \setlength{\itemsep}{0pt}\setlength{\parskip}{0pt}}
\setcounter{secnumdepth}{0}
% Redefines (sub)paragraphs to behave more like sections
\ifx\paragraph\undefined\else
\let\oldparagraph\paragraph
\renewcommand{\paragraph}[1]{\oldparagraph{#1}\mbox{}}
\fi
\ifx\subparagraph\undefined\else
\let\oldsubparagraph\subparagraph
\renewcommand{\subparagraph}[1]{\oldsubparagraph{#1}\mbox{}}
\fi

%%% Use protect on footnotes to avoid problems with footnotes in titles
\let\rmarkdownfootnote\footnote%
\def\footnote{\protect\rmarkdownfootnote}

%%% Change title format to be more compact
\usepackage{titling}

% Create subtitle command for use in maketitle
\providecommand{\subtitle}[1]{
  \posttitle{
    \begin{center}\large#1\end{center}
    }
}

\setlength{\droptitle}{-2em}

  \title{ProyectoRM\_1}
    \pretitle{\vspace{\droptitle}\centering\huge}
  \posttitle{\par}
    \author{Alicia Viñas Zarco}
    \preauthor{\centering\large\emph}
  \postauthor{\par}
    \date{}
    \predate{}\postdate{}
  

\begin{document}
\maketitle

{
\setcounter{tocdepth}{2}
\tableofcontents
}
A lo largo de este documento vamos a llevar a cabo una sencilla revisión
de algunos elementos estudiados durante estas semanas.

Para ello, vamos a dividir el documento en tres secciones.

\begin{enumerate}
\def\labelenumi{\arabic{enumi}.}
\tightlist
\item
  \textbf{Expresiones regulares}
\end{enumerate}

Identificaremos en internet dos páginas donde se discuta sobre el uso de
expresiones regulares en RStudio, analizando cual de las dos es más
completa

\begin{enumerate}
\def\labelenumi{\arabic{enumi}.}
\setcounter{enumi}{1}
\tightlist
\item
  \textbf{Markdown}
\end{enumerate}

Realizaremos el mismo análisis que en la sección anterior, pero con
páginas sobre Markdown.

\begin{enumerate}
\def\labelenumi{\arabic{enumi}.}
\setcounter{enumi}{2}
\tightlist
\item
  \textbf{Estudio descriptivo de Airquality dataset}
\end{enumerate}

En esta última sección trabajaremos con uno de los datasets más
populares: Airquality.

\hypertarget{expresiones-regulares}{%
\section{Expresiones regulares}\label{expresiones-regulares}}

Las expresiones regulares, también conocidas como \emph{regex}, son una
herramienta para la manipulación de cadenas de texto (strings). Es
importante conocer y controlar el uso de esta herramienta, por lo que a
continuación vamos a analizar y comparar dos páginas que hablan sobre
las principales expresiones regulares

\begin{itemize}
\tightlist
\item
  \textbf{\emph{Recurso nº1}}
\end{itemize}

Vamos a comenzar revisando este
\href{http://rpubs.com/ydmarinb/429756}{documento} publicado en
internet, que ha sido realizado también con Rmardown.

Tras una breve definición, abre una amplia sección de \textbf{teoría},
donde se estudian distintos aspectos de las expresiones regulares:

\begin{itemize}
\tightlist
\item
  Cuantificadores
\item
  Alternación
\item
  Agrupación
\item
  Barra invertida
\item
  Signo de admiración
\item
  Metacaracteres especiales
\end{itemize}

Sin embargo, antes de desarrollar estos puntos, el autor define el
concepto \emph{cláusula de Kleene}:

\begin{quote}
Es una operación que se aplica sobre un conjunto de cadenas de
caracteres o un conjunto de símbolos o caracteres, y representa el
conjunto de las cadenas que se pueden formar tomando cualquier número de
cadenas del conjunto inicial, posiblemente con repeticiones, y
concatenándolas entre sí.
\end{quote}

Esta definición puede ser útil para aquellos usuarios que no tengan
conocimientos previos avanzados en matemáticas y/o computación.

Después comienza a revisar cada uno de los aspectos enumerados
anteriormente. Proporciona los casos más comunes dentro de cada
categoría, describiendo su uso.

Como esta página hace referencia específicamente a las expresiones
regulares en RStudio, tiene una sección llamada \textbf{implementación
en R}. Aquí comenta los comandos más utilizados para trabajar con
cadenas de texto:

\begin{longtable}[]{@{}lll@{}}
\toprule
Detección del patrón 1 & Encabezado 2 & Encabezado 3\tabularnewline
\midrule
\endhead
renglón 1, columna 1 & renglón 1, columna 2 & renglón 1, columna
3\tabularnewline
renglón 2, columna 1 & renglón 2, columna 2 & renglón 2, columna
3\tabularnewline
renglón 3, columna 1 & renglón 3, columna 2 & renglón 3, columna
3\tabularnewline
\bottomrule
\end{longtable}

\begin{itemize}
\tightlist
\item
  \textbf{\emph{Recurso nº2}}
\end{itemize}

\url{https://eldesvandejose.com/2017/08/18/expresiones-regulares-general/}

\begin{center}\rule{0.5\linewidth}{\linethickness}\end{center}

fallos - ejemplos por cada cosa

\hypertarget{cuidao}{%
\section{CUIDAO}\label{cuidao}}

Roses are {red}, violets are {blue}.

Cosas guapisimas con mtcars
\url{http://larmarange.github.io/analyse-R/rmarkdown-les-rapports-automatises.html\#en-tete-preambule}

ojo con el tamaño de las gráficas, creo qeu se especifica en la cabecera

\hypertarget{expresiones-regulares-1}{%
\subsection{Expresiones regulares}\label{expresiones-regulares-1}}

Cheat sheets + imagen

\url{https://i.pinimg.com/originals/22/13/ad/2213ad0620372d995a0b9f79ba5215b5.jpg}

ECUACIONEEES

\hypertarget{airquality}{%
\section{airquality}\label{airquality}}

METER ALGUNA CITA Y ASÍ PONGO LO DEL BLOQUE

\url{https://www.rdocumentation.org/packages/datasets/versions/3.6.1/topics/airquality}


\end{document}
